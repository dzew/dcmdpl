\chapter{Conclusions and Future Work}
The way a problem is represented can have a substantial impact on how difficult
it is to solve.
The right representation can make short work of a seemingly complicated
problem.
A good representation is one that is highly correlated with value.
Existing algorithms for representation discovery do not explicitly take value
into consideration and thus cannot discover such a representation in most
situations.
This thesis introduced an iterative representation discovery algorithm that
interleaves steps of value approximation and representation updates.
This algorithm was effective across a range of problems, including three
benchmark reinforcement learning problems.
The work in this thesis opens several exciting avenues for future research.

The first and most natural next step from this thesis is a thorough exploration
of the properties of the FIIRA framework.
It would be interesting to see how different combinations of regressors and
transformers work together.
It would be particularly interesting to see the results of performing FIIRA in
a parametric reinforcement learning setting.
Such an exploration could result in formal proofs about FIIRA's properties.

Another possible next step would be an in-depth study of the applicability of
sustained actions to KBRL.
Sustained actions were instrumental in the experiments detailed in Chapter 5,
especially in the PinBall domain.
Without sustained actions, the number of sample transitions needed to cover
the space would have exceeded the available computational resources.
I firmly believe that there is some fundamental principle behind sustained
actions that makes them worth studying.

Yet another possible direction would be to seek out new applications for FIIRA.
This thesis discusses only reinforcement learning, but FIIRA could be applied
in other function approximation tasks.
The particular FIIRA algorithm described in this thesis could also be used as
a clustering algorithm.
It collapses the domain into a finite set of atoms and has been demonstrated
effective at identifying discontinuities.
This suggests potential as an algorithm for clustering by value.
